Подвижной поворотный кран для ремонта
 уличной электросети установлен на
  автомашине массы 1 т.
Люлька $K$ крана, курепленная на стержне $L$,
 может поворачиваться вокруг
  горизонтальной оси $O$,
перпендикулярной плоскости рисунка. В
 начальный момент кран,
занимавший горизонтальное положение, и
 автомашина находилась
в покое. Определить перемещение
 незаторможенной автомашины, 
если кран повернулся на $60^\circ$. Масса
 однородного стержня $L$ длины
3 м равна 100 кг, а люльки $K$ -- 200 кг.
Центр масс $C$ люльки $K$ отстоит от оси $O$ на
 расстоянии
$OC = 3{,}5$ м. Сопротивлением движению пренебречь.
