Стрелка гальванометра длины $3$ см колеблется вокруг неподвижной оси по закону
$\varphi = \varphi _0 \sin{kt}$.
Определить ускорение конца стрелки в её среднем и крайних положениях,
а также моменты времени, при которых угловая скорость $\omega$ и ускорение
$\varepsilon$ обращаются в нуль, если период колебаний равен $0.4$ с,
а угловая амплитуда $\varphi _0 = \pi/30$.
