Велосипедный трек на кривых участках пути имеет виражи,
профиль которых в поперечном сечении представляет собой прямую,
наклонную к горизонту, так, что на кривых участках
внешний край трека выше внутреннего.
С какой наименьшей и с какой наибольшей скоростью можно ехать по виражу,
имеющему радиус $R$ и угол наклона к горизонту $\alpha$,
если коэффициент трения резиновых шин о грунт трека равен $f$?
