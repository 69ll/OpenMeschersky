К основанию сейсмометра с индукционным преобразователем
прикреплена катушка из $n$ витков радиуса $r$, соединённая с электрической
регистрирующей системой, схематизируемой цепью с самоиндукцией $L$
и сопротивлением $R$.
Магнитный сердечник, создающий радиальное магнитное поле, характеризуемое
в зазоре магнитной индукцией $B$, опирается на основание с помощью пружин
общей жёсткости $c$.
На сердечник действует также сила сопротивления, пропорциональная его скорости,
вызываемая демпфером, создающим силу сопротивления $\beta\dot{x}$.
Составить уравнения, определяющие перемещение сердечника и ток в цепи
в случае малых вертикальных колебаний основания сейсмометра
по закону $\xi = \xi _0\sin{\omega t}$.
