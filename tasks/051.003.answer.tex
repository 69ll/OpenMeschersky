1) $v_1 = \sqrt{\dfrac{\mu}{r}} = \sqrt{\dfrac{gR^2}{R + H}}$ (Угловая скорость на высоте $H$ для данного небесного тела);

2) $T = 2\pi r\sqrt{\dfrac{r}{mu}} = 2\pi\dfrac{(R+H)^{ 3/2 }}{R\sqrt{g}}$. Здесь $r$ - расстояние от материальной точки до центра небесного тела, $/mu$ - его гравитационный параметр, $g$ - ускорение силы тяжести на его поверхности.
