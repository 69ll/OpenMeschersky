 Вращающаяся часть подъемного крана состоит
  из стрелы $CD$
 длины $L$ и массы $M_2$ и груза $K$ массы $M_3$. 
 Рассматривая стрелу как однородную тонкую 
 балку,
 а противовес $E$ и круг $K$ как точечные массы,
 определить момент инерции $J_2$ крана
  относительно
 вертикальной оси вращения $z$ и центробежные
  моменты
 инерции относительно осей коорбинат $x$, $y$, $z$,
 связанных с краном. Центр масс всей системы
  находится
 на оси $z$; стрела $CD$ расположена в плоскости $yz$.
