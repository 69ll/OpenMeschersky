Вращающаяся часть подъёмного крана состоит из стрелы $CD$ длины $L$
и массы $M_1$, противовеса $E$ и груза $K$ массы $M_2$ каждый.
(См. рисунок к задаче $34.31$.)
При включении постоянного тормозного момента кран,
вращаясь до этого c угловой скоростью соответствующей $n = 1.5$ об/мин,
останавливается через $2$ с.

Рассматривая стрелу как однородную тонкую балку, а противовес с грузом
как точечные массы, определить динамические реакции опор $A$ и $B$
крана в конце его торможения.
Расстояние между опорами крана $AB = 3$ м, $M_2 = 5$ т, $M_1 = 8$ т,
$\alpha = 45^{\circ}$, $L = 30$ м, $l = 10$ м,
центр масс всей системы находится на оси вращения;
отклонением груза от плоскости крана пренебречь.
Оси $x$, $y$ связаны с краном.
Стрела $CD$ находится в плоскости $yz$.
%TODO: ссылка на задачу
