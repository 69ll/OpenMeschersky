(Б.Н. Садовский)
Барон Мюнхгаузен обнаружил около звезды $S$ планету $P$ и заметил, что
законы движения Земли вокруг Солнца и $p$ вокруг $S$ в подходящих
инерциальных системах отсчёта имеют вид $r(t)=f(t)$ и $r(t)=f(t)/2$ с
одной и той же функцией $f(t)$. Прав ли был барон, когда сделал из
этого вывод, что масса звезды $S$ в $8$ раз больше массы Солнца?
