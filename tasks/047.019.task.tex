В регуляторе четыре груза одинаковой массы $M_1$ находятся на концах
двух равноплечих рычагов длины $2l$, которые могут вращаться
в плоскости регулятора вокруг конца шпинделя $O$ и образуют с осью
шпинделя переменный угол $\varphi$.
В точке $A$, находящейся от конца шпинделя $O$ на расстоянии $OA = a$,
со шпинделем шарнирно соединены рычаги $AB$ и $AC$ длины $a$,
которые в точках $B$ и $C$ в свою очередь сочленены со стержнями
$BD$ и $CD$ длины $a$, несущими муфту $D$.
В точках $B$ и $C$ имеются ползунки, скользящие вдоль рычагов, несущих грузы.
Масса муфты равна $M_2$.
Регулятор вращается с постоянной угловой скоростью $\omega$.
Найти связь между углом и угловой скоростью $\omega$
в равновесном положении регулятора.
